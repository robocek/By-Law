\chapter{Member Roles and Responsibilities}

\section{Membership}

\subsection{Eligibility}
The club membership will be drawn primarily from the body of full-time students and the alumni in GCE,Kannur. Students with greater involvement with club activities are eligible for membership as per the decision of the execom and staff advisors.

\subsection{Recruitment}
The faculty club advisor, with the consent of the club’s executive committee, determines the potential student and recruits them into the group, the numbers should not exceed a limit which leads to administration unbalance. A nominal fee can be collected as a membership fee to meet the expense of the club. Two list can be maintained: \textbf{\textit{Active members list}} and \textbf{\textit{Passive member list}}. Open call for membership shall be into Passive member list. The execom shall monitor students and their activities from both the list and should renew the active member list annually by producing the list towards staff advisors. Students failing to show active participation shall fall back to Passive member list. Active member list shall be considered as the effective strength of the club.

\subsection{Resignation}
Members who no longer wish to participate in club activities may submit a request to the advisors to withdraw his or her information from the team roster. The advisors must comply when notified provided that the resigning member ceases to participate.

\subsection{Dismissal}
The recruited member can be dismissed ,when she/he becomes a burden to the group and its other members by the executive committee or by the advisors in charge.

\section{Advisors}

\subsection{Positions}
The club shall be overseen by at least five faculty advisors (preferably from various departments) at all times during the academic year. An official lab custodian is appointed for the proper accounting of the finished projects and other equipment. A member from student execom shall be assigned the role of reporting finished projects and other equipment.

THE PRINCIPAL of the Govt college of Engineering, Kannur is suggested for MONITORING AND EVALUATION of the club activities

\subsection{Duties}
The advisors shall serve as the official liaisons between the ROBOCEK and the administration and Associated Student Body. In addition, the advisors are expected to provide guidance to the officers and members of the club, make a reasonable effort to supervise regular group meetings when possible, and be liaison with the school with regard to the team money and finances. The advisors are responsible for determining the eligibility of potential members and mentors and supervising and certifying the results. Finally, the advisors are responsible for taking reasonable steps to address any problems or conflicts of a serious or emergency nature that may arise during club activities and are brought to the attention of the advisor. They have the right to audit any documents (including financial) pertaining to any activities initiated by the Students Association. The club bank account shall be jointly held by a staff advisor and treasurer of Student execom.

\subsection{Term Length}
Under normal circumstances, advisors serve for the duration of the academic year and retain their position carrying forward into the following  academic year.

\subsection{Resignation}
An advisor may choose to resign from his or her position at any time; however, he or she is highly encouraged to continue to serve, if at all possible, until a replacement is found.

\subsection{Dismissal}
In the unlikely event that an advisor demonstrates a long-term or severe failure to act according to club policies or perform the expected duties of his or her position, the elected student may, by majority vote, elect to recommend the removal of the advisor to the club. Student officers will need to communicate and work with THE PRINCIPAL Govt college of Engineering, Kannur to facilitate the removal of an advisor.

\subsection{Appointment}
At any time during the academic year, the elected officers may recommend a candidate to become an additional advisor. An all club meeting must be held to vote on approval of the new advisor, where a majority vote of present members is needed with a majority of active club members and mentors present for the advisor to be confirmed.

\section{Student Advisory Committee}
\subsection{Purpose}
Student advisors empower students' knowledge in science and technology by sharing their views and ideas . As such, the club has to actively recruit or keep in contact with industry mentors and other experienced adults to attend club meetings and provide advice throughout all stages of team projects.

\subsection{Eligibility}
Previous execom members and/or alumni members are preferred to be the student advisory committee to provide guidance for the newly elected student body.

\subsection{Duties}
Student advisors must agree to be receptive and encouraging to students to help foster an environment that is conducive to student learning. While they are expected to participate in team activities and offer guidance and advice, the majority of hands-on work is to be completed by students. Student advisors must also agree to provide basic contact information for registration and communication purposes 

\subsection{Respect and Recognition}
Student Advisors shall be treated with the utmost respect and gracious professionalism by all team members. If there is an occasion where a team member does not treat a student advisor with respect and gracious professionalism, it should be reported to \textbf{the club advisor and the club president}. They have to take necessary actions. Student advisors will be thoughtfully recognized by team members and elected leadership at the end of each academic year.

\subsection{Resignation}
An advisor may choose to discontinue his or her participation by communicating his or her resignation to the staff advisors. Upon doing so, the student advisor shall be removed from the list of registered advisory committees.

\subsection{Dismissal}
In the unlikely event that a student advisor fails to meet the above duties and guidelines, grossly interferes with the team’s intent to implement the projects , does not employ gracious professionalism with team members, the advisors, and other mentors, the staff advisor should be notified and the staff advisor will talk with him/her about these issues and the incident. 

\section{Office bearers}

\subsection{Positions}
The main governing body of the ROBOCEK shall be its student office bearers, chosen from various engineering branches to promote equity in participation and decision making in all aspects. The officer positions are as follows:\\

President, Vice President, Secretary, Treasurer, Project Management Head, Chief Technical Lead, Social Media Strategist, Design Lead, Outreach Lead and the Alumni Link.

\subsection{Eligibility and Term Length}
Officers must be eligible members of ROBOCEK at the time of their election and for the duration of the academic year in which they serve. No member may fill more than one of the positions. Offices are held from the point of election until the point of the next election.

\subsection{Duties}
The officers are collectively responsible for the management of the Group. In addition, each officer has the following individual responsibilities: 

\begin{enumerate}
	\item The \textbf{President} is the chief student coordinator who oversees the operations of the club as a whole. Responsibilities include but are not limited to the following: working with the leadership group to plan and lead team meetings and events, setting team deadlines and ensuring that they are met, communicating with the advisors , other officers, and any appointed leads to coordinate and organize team project efforts, recruiting and communicating with mentors developing and maintaining relationships with outside groups. 
	
	\item The \textbf{Vice President} is the student who needs to assist the president in all his activities and in the absence of the president, it is the duty of the vice president to monitor club activities. He/She (along with the secretary) should ensure that proper records are maintained which includes but not limited to, financial records, component and equipment list, active member list, event and program reports etc.
	
	\item The \textbf{Secretary} is the student who is responsible for the proper keeping of all documents of the club, which includes but not limited to event reports, program schedules and activities, meeting minutes etc. He/She is the one to conduct meetings of the club and to note the important decisions. He/She must also ensure that every decision taken is put into practice and also keep the important credentials including passwords and login information.
	
	\item The \textbf{Treasurer} is the student who is responsible for the fund management (treasury) of the club and to keep a full fledged account book for the club. He/she is responsible for handling and allocating the funds for the club and shall monitor bank accounts if any. The account book shall be verified by the staff advisor as and when necessary at any time. The club bank account shall be jointly maintained by Treasurer and Staff advisor.
	
	\item The \textbf{Project Management Head} is responsible for planning, scheduling, and managing resources of team projects. He/she must frequently communicate with the other officers and any appointed leads throughout the project build to make sure that the projects run on time, facilitate design changes, analyze costs etc. The Head shall also maintain active projects, and encourage student members to engage in new projects and build skills of their interest. He/she along with the \textbf{lab custodian} shall be responsible for maintaining the list of components.
	
	\item The \textbf{Chief Technical Lead} is responsible for the technical aspects including maintenance of ROBOCEK server, Official website and provides technical resources as and when required. Provide the needy guidance in technical matters pertaining to the activities of the club. The Lead shall also mentor Tire II members to build necessary skills to handle above mentioned domains. Technical resources include, but not limited to, resource materials, guidance/training of relevant skills.
	
	\item The \textbf{Social Media Strategist} is responsible for the publicity of the club activities through social media handles. Also should adopt timely measures to keep the social media pages lively and engaged. Social media includes, Instagram, LinkedIn, Youtube channel etc showing social presence. The account credentials shall be jointly handled by Social Media Strategist and \textbf{Secretary}.
	
	\item The \textbf{Design Lead} is responsible for the quality and creativity of the design aspects related to activities of the club, including posters, write-ups and brochures. The lead should also engage in activities that promote quality and timely delivery of contents. They can include, but not limited to, training of subordinates (Tire II) to ensure that valid design skills are built among interested members.
	
	\item The \textbf{Outreach Lead} is responsible for formulating plans and leads outreach programs of the club. Devise timely strategies to strengthen the connections of the club with external entities. The lead can work in conjunction with Social Media Strategist and Alumni Link to obtain his/her objectives. Few outreach activities include, but not limited to, out-sourcing projects (at the discretion of the execom), engaging the training activities out of Institution, inviting tech-expo programs, etc. The lead shall promote the social image of the club.
	
	\item The \textbf{Alumni Link} is responsible for building healthy connection links with the ROBOCEK alumni. The lead shall update alumni with relevant decisions made in the student's executive body. The lead may be added to Alumni execom, serving as an active link between students, faculty and Alumni. With decisions taken from alumni execom, relevant programs/interactions can be planned with student members of the club.
	
\end{enumerate}
The above roles shall be considered as overview of responsibility. The execom should work hand-in-hand to achieve the mission and goals of the club. They should jointly taking into account of Tire II members by recruiting, training, and developing strong bond among themselves and the technical community.   

\subsection{Resignation}
An officer may choose to resign from his or her position by submitting a formal letter of resignation to the advisors. The position will remain vacant until a special or general election is scheduled and held.

\subsection{Dismissal}
In the event that an elected officer, lead, or captain demonstrates a long-term or severe failure to act according to club policies or perform the expected duties of his or her position, the remaining elected student officers and advisors may, by majority vote, elect to remove him or her from office. If the officer is dismissed, the position will remain vacant until a special election is scheduled and held.